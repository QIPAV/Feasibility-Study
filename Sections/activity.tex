\section{An activity layout of the project}

In this chapter you will be provided with a very rough activity list of the activities we expect to face during this project. For a more detailed version check \textbf{figure \ref{list}}. This is only a preliminary version, of how we assume the project will go at the current stage. We will somewhere along the project encounter unexpected problems and this activity layout will change. All complex projects will run into unforeseen challenges. However by using a project model we can plan for complex projects. With Scrum we can lower the risks of the project, creates consistency and control of change and control of development. As long as we do this right we will stand a chance at managing the time, and hopefully not end up with too many changes to the preliminary plan.\\
\\
This first activity in the project is a Start-Up Phase. In this project we are using Scrum, but since we have little experience with this we will need some time to research and document how we will use this project model. The first activity then would naturally be a Start-Up Phase. However before we start the sprints, we will require some research and planning. First of all we need to understand the problem, and what we are going to solve. A solution without the knowledge of what exactly the problem is, will in almost all cases be a bad solution. Knowing what is asked of you, is key. \\ 
\\
The planning phase will consist of everything from research to construction templates for documents so we can minimize work redundancy. The advantage with this is that we can decrease development time a bit. The time we spend in the planning phase will however increase. But it is hard to argue that getting more information from the start will not impact how we design the system. It's important when designing a project that you spend time in the planning phase. The more time spent in the planning phase tend reduce errors further in the project, and sometimes can even save the project from catastrophe. Going through different concepts on sub-components and analyzing the advantages and disadvantages of these will probably improve the quality of the end-product. This phase of creating a system can really not be over-emphasized. \\
\\
Another activity is to do an investigation around system planning. How we would develop the system. Refreshing some knowledge in system engineering will undoubtedly help with improved products, better design and higher overall quality.
The activities would then be divided into sprints, where each sprint contains the phases evaluation, requirements, design and analysis, implementation, testing and deployment. The reason all those phases is included is to develop a potentially shippable product every sprint. Which increases the feedback from the customer and gives the developers valuable information. The information can then be used to make a better product for each sprint. \\
\\
Our first sprint will contain mostly of planning but also a lot of documentation and preliminary studies. The second sprint contains a preliminary construction, preliminary flight controller and electrical layout. Compared to the first sprint, the second sprint will be the first where we actually get a product, drawings and ideas on how the quadcopters will look. \\
\\
According to the current activity layout, we should hopefully have a working prototype quadcopter to test with by the end of the third sprint. We should also start working on the home-made flight controller and try to manage take-off and landing. The mechanical design of the quadcopter should be continued to be improved upon, but in order to keep our schedule, we will need something to test with as soon as possible. \\
\\
In sprint 4 and 5 we will be working on adjusting the flight controller and include advanced flight, we will also need to review the mechanical design. 
In Sprint 6 we will continue the work to finalize the test procedures and tweak the flight controller. By the end of sprint 7 we will hopefully have a a working flight controller so that we can use the remaining sprints on testing and comparison between fixed and variable pitch. 
Sprint 8 and 9 will then contain the remaining testing and documentation required in the project. The last sprint will however only be a week, making our deadline for the project around the 16th of may. 
\textbf{Figure \ref{list}} displays the activities and their rough estimated deadlines. It also lists which main milestones we want to achieve for each sprint. 
\\

\begin{center}
\section*{\textbf{Activity Layout}}
\begin{tabular}{cllll}
\rowcolor{cadetgrey}
\textbf{ID:}    &\textbf{Activity:} 	 &\textbf{Estimated time:}    &\textbf{Start:}  \\ % &\textbf{Main responsibility:}  \\
    
\textbf{1} & \textbf{Start-Up Phase} & \textbf{260h} & $03.01.17$ 
\\\rowcolor{gainsboro}
1.1       & Feasibility Study     &     &  \\
1.2       & Research     &     &  \\\rowcolor{gainsboro}
1.3       & Project Model     &     & \\
1.4       & Project Plan     &     & 
\\\rowcolor{gainsboro}
1.5       & Requirements     &     & \\
1.6       & Test Specification     &     & 
\\\rowcolor{gainsboro}
1.7       & Document Templates     &    & \\
\textbf{2} & \textbf{Presentation 1}     & \textbf{100h}    & $03.01.17$ 
\\\rowcolor{gainsboro}
2.1     & Preliminary Documentation Refinement  &    & \\
2.2     & Presentation Refinement  &    &
\\\rowcolor{gainsboro}
2.3     & Presentation Practice  &    & \\
\textbf{3} & \textbf{Sprint 1}     & \textbf{360h}     & $17.01.17$ 
\\\rowcolor{gainsboro}
3.1     & First Presentation &  & \\
3.2     & Preliminary Flight Simulation &  & \\\rowcolor{gainsboro}
3.3     & Communication With Qualisys &  & \\
3.4     & Mechanical Design Study &  & 
\\\rowcolor{gainsboro}
\textbf{4} & \textbf{Sprint 2}     & \textbf{360h}     & $31.01.17$ \\
4.1     & Electrical Analysis &  &  \\\rowcolor{gainsboro}
4.2     & Electrical Layout &  & \\
4.3     & Electrical Construction & & \\ \rowcolor{gainsboro}
4.4     & Flight Controller & & \\
4.5     & Create Web Page & &  
\\\rowcolor{gainsboro}
4.6     & Mechanical Design & & \\
4.7     & Preliminary Construction & & 
\\\rowcolor{gainsboro}
\textbf{5} & \textbf{Sprint 3}     & \textbf{360h}     & $14.02.17$ \\
5.1     & Plan Test Procedures &  &  \\\rowcolor{gainsboro}
5.2     & Mechanical Design &  &  \\
5.3     & Preliminary Variable Pitch Design & & \\ \rowcolor{gainsboro}
5.4     & Flight Testing and Controll &  & \\
\textbf{6} & \textbf{Sprint 4}     & \textbf{360h}     & $28.02.17$ \\
\rowcolor{gainsboro}
6.1     & Mechanical Design Review & & \\
6.2     & Tweak Flight Controller & & \\
\rowcolor{gainsboro}
\textbf{7} & \textbf{Sprint 5}     & \textbf{360h}     & $14.03.17$ \\
7.1     & Advanced Controll Functionality &  & \\\rowcolor{gainsboro}
\textbf{8} & \textbf{Presentation 2}     & \textbf{80h}     & $28.03.17$ \\
8.1     & Presentation 2 planning &  &  \\\rowcolor{gainsboro}
\textbf{9} & \textbf{Sprint 6}     & \textbf{360h}     & $03.01.17$ \\
9.1     & Finalize Test Procedures &  & 
\\\rowcolor{gainsboro}
9.2     & Tweak Controller System &  & \\
\textbf{10} & \textbf{Sprint 7}     & \textbf{360h}     & $11.04.17$ \\
\rowcolor{gainsboro}
10.1     & Final Flight Controller Adjustment &  & \\
10.2     & Testing and Comparison & & \\
\end{tabular}                                                                   
\end{center}
\vfill

\begin{center}
\section*{\textbf{Activity Layout}}
\begin{tabular}{cllll}
\rowcolor{cadetgrey}
\textbf{ID:}    &\textbf{Activity:} 	 &\textbf{Estimated time:}    &\textbf{Start:}  \\ % &\textbf{Main responsibility:}  \\

\rowcolor{gainsboro}
\textbf{11} & \textbf{Sprint 8}     & \textbf{360h}    & $25.04.17$ \\
11.1     & Testing and Comparison              &  &  \\\rowcolor{gainsboro}
11.2     & Documentation &  &  \\
11.3     & Testing & &  \\ 
\rowcolor{gainsboro}
\textbf{12} & \textbf{Sprint 9}     & \textbf{180h}     & $09.05.17$ \\
12.1     & Testing and Comparison &  &  \\\rowcolor{gainsboro}
12.2     & Documentation Review &  & \\
\textbf{13} & \textbf{Presentation 3}     & \textbf{360h}     & $09.05.17$ \\
\rowcolor{gainsboro}
13.1     & Documentation Review &  & \\
13.2     & Practice & & \\
\rowcolor{gainsboro}
         & \textbf{Total amount of hours estimated:} & \textbf{3240h} & 
\end{tabular}                                                               
\end{center}
\begin{figure}[h]
        \centering
          \centering
            \caption{Activity Layout}
            \label{list}
\end{figure}